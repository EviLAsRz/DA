ALGORITMO DE FUSION O MERGESORT.


Especificación del algoritmo:

-Algoritmo Divide y Vencerás perteneciente al esquema de Equilibrados.

-Recibe un (std::vector<t> &arr, size_t l, size_t r), donde t y r representa el rango desde 0 hasta tam -1.  
-Precondiciones: Elementos de la lista deben tener sobrecargado el operador < y el <=.
-Postcondiciones: Devuelve la lista de t elementos ordenados.

Al igual que la versión vista en teoría, el algoritmo utiliza tres parámetros:
-i representa la posición a partir de la cual vamos a ordenar el vector.
-j representa la posición tope de la ordenación del vector, es decir, hasta donde se va a ordenar.
-k representa la posición intermedia por la dividiremos el vector.

template <class t>
void mergeSort(std::vector<t> &arr,size_t l,size_t r){
    if(l>=r){
        return;
    }
    size_t m = (l+r-1)/2;
    mergeSort(arr,l,m);
    mergeSort(arr,m+1,r);
    merge(arr,l,m,r);
}

template <class t>
void merge(std::vector<t> &arr, size_t l, size_t m, size_t r)
{
    size_t n1 = m - l + 1;
    size_t n2 = r - m;
        
    std::vector<t>L(n1),R(n2);
   
    for (int i = 0; i < n1; i++){//copia la parte izquierda al centro
        L[i] = arr[l + i];
    }
        
    for (int j = 0; j < n2; j++){//copia la parte derecha al centro
         R[j] = arr[m + 1 + j];
    }

    int i = 0;
    int j = 0;
    int k = l;
 
    while (i < n1 && j < n2) {
        if (L[i] <= R[j]) {
            arr[k] = L[i];
            i++;
        }
        else {
            arr[k] = R[j];
            j++;
        }
        k++;
    }
    while (i < n1) {
        arr[k] = L[i];
        i++;
        k++;
    } 
    while (j < n2) {
        arr[k] = R[j];
        j++;
        k++;
    }
}