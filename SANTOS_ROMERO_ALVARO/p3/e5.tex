Analisis temporal de los algoritmos de ordenación de la practica:

Algoritmo de Fusion:

Siendo un tablero, sabemos que tenemos f filas y c columnas.
Una celda se compone de un elemento fila y un elemento columna, por lo que
n = f X c

Cuando n es menor que el umbral establecido, se van a ejecutar n(n-1)/2 operaciones patrón (orden O(n))
Para tamaños de n mayor que el umbral, se hacen dos llamadas recursivas de tamaños (n/2) aproximadamente.
Por lo que, para este caso, se hacen dos ordenaciones de (n/2).

Resumiendo:

        
        [ n(n-1)/2,                                   n <= n0                         
t(n) => [                                                    
        [ t([por defecto]n/2) + t([por exceso]n/2),   n > n0 

Resolviendo la ecuación de recurrencia, observamos que en el peor caso, el orden será de O(n * log n)

Para la ordenación de las defensas, se utiliza un algoritmo devorador que tendrá un orden O (n^2), pero
como se realiza una preordenación de las celdas, el orden cambia a (n * log n).

En conclusión, el t(n) para el algoritmo de ordenación de las defensas será = t(n * log n) + t (n * log n)

Algoritmo de Ordenación Rápida:

Siendo un tablero, sabemos que tenemos f filas y c columnas.
Una celda se compone de un elemento fila y un elemento columna, por lo que
n = f X c

EL algoritmo consta de dos partes, la función de partición que ordena los elementos hasta el pivote y
la función de ordenación que se llama recursivamente con las dos mitades del vector. 
Para vectores suficientemente pequeños, el t(n) será de n(n-1)/2.
Para vectores mayores que n, se hacen dos llamadas recursivas con t(n) = t(n/2) + n

Resumiendo:

        
        [ n(n-1)/2,                                       n <= n0                          
t(n) => [                                                    
        [ t([por defecto]n/2) + t([por exceso]n/2) + n,   n > n0 

Resolviendo la ecuación de recurrencia, observamos que en el peor caso, el orden será de O(n * log n) y
al igual que en el algoritmo de fusión, este será de orden O 2(n * log n) al tener que ordenar las defensas también (s
utiliza el mismo algoritmo devorador en ambos casos).


En resumen, estos algoritmos tienen el mismo orden, difiriendo en la constante multiplicativa.
