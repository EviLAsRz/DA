Para realizar las pruebas de caja negra, vamos a utilizar un par vectores opuestos, de tal
forma que al utilizar alguno de los algoritmos de ordenación, compararemos que el resultado coincide
con el opuesto.

bool prueba_vector_estatico(){
std::vector<int>v = {5,4,3,2,1};
std::vector<int>v2 = {1,2,3,4,5};
unsigned op;
bool ordenado = true;
std::cout<<"1-.Ordenacion por fusion\n2-.Ordenacion rapida"<<std::endl;
std::cin>>op;
switch(op){
    case 1: mergeSort(v,0,v.size()-1);
    break;
    case 2: quickSort(v,0,v.size()-1);
        break;
    }
    //Comparación elemento a elemento
     for(size_t i=0; i<v.size();i++){
        if(v[i]!= v2[i]){
            ordenado = false;
        }
    }

    return ordenado; //true si esta ordenado
}

Para realizar la prueba de caja negra de vectores pseudodinámicos, utilizaremos vectores de tamaño 100 hasta 10000, con
aumentos de 100 en 100 (tamaños 100,200,300,etc).
Devuelve true si después de la ejecución de los algoritmos, estos vectores están ordenados.

    std::vector<int>v,u;
    bool ordenado = true;
    unsigned op;
    size_t t = 100; //tamaño inicial del vector
    
    while(t < 10000 && ordenado){ //mientras que no supere el limite y ordenado sea true
        //se inicializa el vector con el tamaño
        v = std::vector<int>(t);
        for(size_t i = 0; i < t; i++)
        v[i] = i+1;
    
    u = v;  //a priori, u ya está ordenado de menor a mayor
    //hacemos un shuffle para desordenar el vector usando random_device
    std::shuffle(v.begin(),v.end(), std::random_device());

    //llamamos a ordencion fusion o rapida
    //mergeSort(v,0,v.size()-1);   
    quickSort(v,0,v.size()-1);
    
    //se comprueba que está ordenado con u que tiene el vector ordenado decrecientemente
    for(size_t i=0; i<v.size();i++){
        if(v[i]!= u[i]){
            ordenado = false;
        }
    }
        //se aumenta el tamaño del vector en 100
        t+=100;
    }
    return ordenado;
}