El algoritmo a utilizar será el algoritmo A*.

Este algoritmo nos permite encotrar soluciones mediante las comparaciones entre nodos.
La función que representa el algoritmo el: f(n) = h(n) + c(n).

Explicación de los valores:

->h(n): Representa el valor heurístico del nodo,en mi caso voy a utilizar la distancia de
manhattan.

->c(n): Representa el coste de movernos de un nodo A a un nodo B.

Funcionamiento del algortimo:

En A*, tenemos tres variables principales, que son actual (iterador que almacena el nodo actual),
abiertos (lista que almacena los nodos abiertos y explorables) y cerrados (lista que almacena los nodos ya visitados)

Para un correcto funcionamiento del algoritmo, se ha ordenado la lista de abiertos para reducir el tiempo necesario de
encontrar una solución.
Para ello, se ha utilizado la clase montículo de std.

Para la recuperación del algoritmo A*, se utilizará una función que reciba el nodo inicial y el nodo final.